\section{Introduzione}


\subsection{Scopo del progetto}
Lo scopo del progetto è la progettazione e l'implementazione di
un'applicazione per la risoluzione di CP-Net che permetta di risolvere
una CP-Net e di visualizzare il grafo rappresentante l'ordine di tutti
i possibili assegnamenti di valori alle variabili del problema.  Una
caratteristica importante dell'applicazione è che risolva CP-Net
cicliche e CP-Net acicliche.

\subsection{Funzionamento generale}
Dato che l'applicazione permette la risoluzione di CP-Net cicliche, è
necessario che la CP-Net in input sia convertita in un problema di
vincoli classico (CSP), quindi il funzionamento generale è il
seguente:
\begin{itemize}
\item si accetta in input una CP-Net rappresentata attraverso una
  grammatica che verrà definita in sezione \ref{sect:file},
\item la CP-Net viene convertita in un CSP,
\item si calcolano e si visualizzano tutte le soluzioni ottime del
  CSP,
\item si genera e si visualizza il grafo dell'ordine degli
  assegnamenti.
\end{itemize}

\subsection{Manuale d'uso}

All'apertura del programma non viene caricata alcuna CP-Net.  In
ciascun momento, attraverso il menu, l'utente può decidere di:
\begin{itemize}

\item \textbf{aprire una CP-Net:} utilizzando il menu \textit{File
    $\rightarrow$ Import From File} e selezionando un file di testo
  formattato come descritto in sezione \ref{sect:file}. A questo punto
  l'applicazione visualizzerà le soluzioni e il grafo degli ordini di
  tutti i possibili assegnamenti.

\item \textbf{modificare la CP-Net che si sta visualizzando:}
  utilizzando il menu \textit{File $\rightarrow$ Edit current CP-Net},
  sempre con la sintassi descritta in sezione \ref{sect:file}. Le
  soluzioni e il grafo degli ordini parziali delle soluzioni
  visualizzati verranno automaticamente aggiornati. Questa funzione
  permette anche di definire una CP-Net completamente nuova senza
  utilizzare alcun file.

\end{itemize}

\subsubsection{Descrizione di una CP-Net}
\label{sect:file}
Il formato usato per la descrizione di una CP-Net è in seguito
descritto.  Per ciascuna variabile Z che dipende dalle variabili X e Y
occorre specificare, nel seguente ordine:
\begin{itemize}
\item il nome della variabile e i nomi delle variabili da cui essa
  dipende, con una sintassi come nel seguente esempio: \texttt{var Z
    dependsOn=\{X,Y\}}. Se la variabile in questione non dipende da
  alcuna altra variabile, la sintassi da usare è la seguente:
  \texttt{var Z dependsOn=\{\}}
\item gli elementi che fanno parte del dominio della variabile, con
  una sintassi come nel seguente esempio: \texttt{dom=\{z, !z\}}.
\item l'ordine associato a ciascun possibile assegnamento delle
  variabili di dipendenza facendo seguire all'assegnamento delle
  variabili di dipendenza il carattere `:' e l'ordine di tutti gli
  elementi del dominio della variabile. Ad esempio, se fosse
  $dom(X)=\{x,!x\}$ e $dom(Y)=\{y,!y\}$ una possibile definizione
  degli ordini di Z è la seguente:
  \begin{verbatim}
    x,y:z>!z
    x,!y:!z>z
    !x,y:z>!z
    !x,!y:!z>!z
\end{verbatim}
  Se la variabile in questione non dipende da alcuna altra variabile è
  sufficiente indicare dopo il carattere `:' il solo ordine di
  preferenza dei valori della variabile come nel seguente esempio:
  \texttt{:z>!z} (o \texttt{:!z>z}).
\end{itemize}

Segue un esempio completo che descrive una CP-Net:
\begin{verbatim}
var X dependsOn={}
dom={x,!x}
:x>!x

var Y dependsOn={}
dom={y,!y}
:y>!y

var Z dependsOn={X,Y}
dom={z,!z}
x,y:z>!z
x,!y:!z>z
!x,y:!z>z
!x,!y:!z>z
\end{verbatim}