\section{Introduzione}


\subsection{Scopo del progetto}
Lo scopo del progetto è la progettazione e l'implementazione di un'applicazione
per la risoluzione di CP-Net, che permetta di visualizzare
attraverso una GUI la CP-Net iniziale e dell'ordine parziale delle soluzioni della
medesima.
Una caratteristica importante dell'applicazione è che permette la risoluzione di
CP-Net cicliche.

\subsection{Funzionamento generale}
Visto che l'applicazione permette la risoluzione di CP-Net cicliche è
necessario che la CP-Net in input sia convertita in un problema di vincoli
classico (CSP), quindi il funzionamento generale è il seguente:
\begin{itemize}
  \item si accetta in input una CP-Net,
  \item la CP-Net viene convertita in un CSP,
  \item si calcolano e si visualizzano tutte le soluzioni ottime del CSP,
  \item si genera e si visualizza il grafo degli ordini parziali delle soluzioni.
\end{itemize}

\subsection{Manuale d'uso}

All'apertura del programma non viene caricata alcuna CP-Net.
In ciascun momento, attraverso il menu, l'utente può decidere di:
\begin{itemize}

\item aprire una CP-Net: utilizzando il menu \textit{File -> Import From File} e
selezionando un file di testo formattato come descritto in sezione 
\ref{sect:file}. A questo punto l'applicazione visualizzerà le soluzioni e il grafo degli ordini parziali delle soluzioni della CP-Net.

\item modificare la CP-Net che si sta visualizzando: utilizzando il menu
\textit{File -> Edit current CP-Net}, sempre attinendosi alla sintassi descritta
in sezione \ref{sect:file}. Le soluzioni e il grafo degli ordini parziali delle
soluzioni visualizzati verranno automaticamente aggiornati.

\end{itemize}

\subsubsection{Descrizione di una CP-Net}
\label{sect:file}
Il formato usato per la descrizione di una CP-Net è in seguito descritto.
Per ciascuna variabile Z che dipende dalle variabili X e Y occorre specificare,
nel seguente ordine:
\begin{itemize}
  \item il nome della variabile e i nomi delle variabili da cui essa dipende, con
  una sintassi come nel seguente esempio: \texttt{var Z dependsOn=\{X,Y\}}. Se la
  variabile in questione non dipende da alcuna altra variabile la sintassi da
  usare è la seguente: \texttt{var Z dependsOn=\{\}}
  \item gli elementi che fanno partedel dominio della variabile, con una sintassi
  come nel seguente esempio: \texttt{dom=\{z, !z\}}.
  \item l'ordine associato a ciascun possibile assegnamento delle variabile di
  dipendenza. Ad esempio, se $dom(X)={x,!x}$ e $dom(Y)={y,!y}$ un possibile
  insieme di ordini è il seguente:
  \begin{verbatim}
    x,y:z>!z
    x,!y:!z>z
    !x,y:z>!z
    !x,!y:!z>!z
  \end{verbatim}
  Se la variabile in questione non dipende da alcuna altra variabile la sintassi
  da usare è la seguente: \texttt{:z>!z} (o \texttt{:!z>z}).
\end{itemize} 

Segue un esempio completo che descrive una CP-Net:
\begin{verbatim}
var X dependsOn={}
dom={x,!x}
:x>!x

var Y dependsOn={}
dom={y,!y}
:y>!y

var Z dependsOn={X,Y}
dom={z,!z}
x,y:z>!z
x,!y:!z>z
!x,y:!z>z
!x,!y:!z>z
\end{verbatim}