\subsection{Trasformazione da CP-Net a CSP}
Come accennato in precedenza, la ragione per cui si ha la necessità di
trasformare la CP-Net in un CSP è che l'applicazione permette la
risoluzione di CP-Net cicliche e per la risoluzione di questo tipo di
CP-Net è necessario effettuare tale trasformazione.

Data una CP-Net è sempre possibile costruire un CSP classico con lo
stesso insieme di soluzioni ottime nel seguente modo: per ogni ordine
$X_1=x_1, ..., X_n=x_n: Y=y_1 > Y=y_2 > \dots > Y=y_k$ si aggiunge al
vincolo definito sulle variabili $(X_1, \dots, X_n, Y)$ la tupla
$(x_1, \dots, x_n, y_1)$.

Vista la rappresentazione adottata per gli ordini, questo corrisponde
a creare, per ciascuna variabile $v$, un vincolo per ogni foglia
dell'albero degli ordini di $v$.  Infatti, considerato che per ogni
foglia si conosce l'istanziazione delle variabili $X_1=x_1, ...,
X_n=x_n$, che è rappresentata da tutti i nodi interni, e la foglia con
il relativo ordine $Y=y_1 > Y=y_2 > ... > Y=y_k$, è sufficiente
applicare il procedimento di aggiunta precedentemente descritto su
tutti i nodi foglia dell'albero per ottenere un vincolo completo.
