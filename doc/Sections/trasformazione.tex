\subsection{Trasformazione da CP-Net a CSP}
Come accennato in precedenza, la ragione per cui si ha la necessità di
trasformare la CP-Net in un CSP è che l'applicazione permette la risoluzione di
CP-Net cicliche, e per la risoluzione di questo tipo di CP-Net è necessario
effettuare tale trasformazione.

Data una CP-Net è sempre possibile costruire un CSP classico con lo stesso
insieme di soluzioni ottime nel seguento modo: per ogni ordine $X_1=x_1, ...,
X_n=x_n: Y=y_1 > Y=y_2 > ... > Y=y_k$ costruiamo il vincolo $X_1=x_1, ..., X_n=x_n
=> Y=y_1$.

Vista la rappresentazione adottata per gli ordini, questo corrisponde a creare, per
ciascuna variabile $v$, un vincolo per ogni foglia dell'albero degli ordini di $v$.
Infatti, considerato che per ogni foglia si conosce l'istanziazione delle variabili
$X_1=x_1, ..., X_n=x_n$ che è rappresentata da tutti i nodi tra la radice e la
foglia, e il relativo ordine $Y=y_1 > Y=y_2 > ... > Y=y_k$
è possibile costruire il vincolo $X_1=x_1, ..., X_n=x_n => Y=y_1$.
