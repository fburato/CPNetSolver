\subsection{Vincoli}
I vincoli sono rappresentati dalla classe \texttt{Constraint},
caratterizzata dai due seguenti campi dati:
\begin{itemize}
\item \texttt{vars: Array[String]} che rappresenta una lista ordinata
  di variabili,
\item \texttt{accepted: Buffer[Array[String]]} che rappresenta la
  lista di tutte le istanziazioni accettate per quelle variabili.
\end{itemize}

% $O(|dom(V_1)| \cdot \dots \cdot |dom(V_n)|)$ ove $ V_1, \dots, V_n $

Ad esempio, date tre variabili $X,\ Y$ e $Z$ tali che
$dom(X)=\{0,1\},\ dom(Y)=\{0,1\}$ e $dom(Z)=\{0,1\}$, l'oggetto di
tipo \texttt{Constraint} che rappresenta il vincolo $X+Y=Z$ avrà
\begin{itemize}
\item \texttt{vars = [X, Y, Z]} e
\item \texttt{accepted = [ [0, 0, 0], [1, 0, 1], [0, 1, 1] ]}.
\end{itemize}

Le operazioni caratteristiche di un vincolo sono:
\begin{itemize}
\item \texttt{projection(variable: String): Set[String]}: corrisponde
  ad un operazione di proiezione rispetto ad una variabile. Tale
  metodo consente di ottenere a partire da un oggetto
  \texttt{Constraint} e fornendo una variabile passata come stringa
  l'insieme di tutti i possibili valori che tale variabile può
  assumere secondo le tuple ammesse dal vincolo. Se ad esempio il
  vincolo definito su $\{X,Y,Z\}$ ammettesse le tuple
  $\{(0,0,1),(0,1,1),(0,1,0)\}$ allora la proiezione rispetto a $X$
  restituirebbe l'insieme $\{0\}$ in quanto il solo valore ammesso per
  $X$ è $0$ in tutte le tuple, mentre la proiezione rispetto a $Y$ e
  $Z$ restituirebbe l'insieme $\{0,1\}$ in quanto, nel complesso,
  tutti questi valori sono ammessi tra le varie tuple.
\item \texttt{reduction(variable: String, value: String): Constraint}:
  corrisponde all'operazione di riduzione ad altro vincolo ammettendo
  che una variabile assuma un certo valore. Il vincolo restituito da
  una riduzione eseguita su una variabile $x$ ad un valore $v$ sarà il
  vincolo di invocazione in cui sono state mantenute tutte e sole le
  tuple che hanno il valore $v$ assegnato alla variabile $x$.
\end{itemize}

Tali operazioni sono fondamentali per la risoluzione del CSP ottenuto
a partire dalla definizione iniziale della CP-Net.
