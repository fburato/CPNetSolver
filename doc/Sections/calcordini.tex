\subsection{Calcolo diagramma di Hässe per gli assegnamenti}
La realizzazione del diagramma di rappresentazione dell'ordine degli assegnamenti necessita innanzitutto di una libreria in grado di gestire e
rappresentare grafi all'interno di un'interfaccia grafica. La scelta è
quindi ricaduta sul ``Java Universal Network/Graph
Framework''\footnote{\url{http://jung.sourceforge.net/}}, una libreria
open source che possiede tutte le caratteristiche necessarie al
progetto.

Una volta che i grafi possono essere rappresentati è necessario
indivduare un modo efficiente per riuscire a calcolare l'ordine degli assegnamenti utilizzando la semantica dei flip peggiorativi. A questo
scopo è stato realizzato un algortimo che è in grado di calcolare in
tempo ammortizzato $O(e + v)$ l'intero diagramma degli ordini, dove
$e$ è il numero di archi del grafo e $v$ il numero di vertici.

\subsubsection{Contatore variabile}
L'astrazione che consente di raggiungere un alta efficienza è quella
del \textit{contatore variabile}. Se un contatore binario è un
contatore in cui tutte gli elementi del contatore lavorano in modulo
2\footnote{Si veda Cormen, Leiserson, Rivest e Stein ``Introduzione
  agli algortimi e Strutture Dati'' - McGraw Hill}, definiamo un
contatore variabile un contatore in cui a ciascun elemento è associato
un diverso insieme quoziente (non solo 2).

Per fare un esempio, in un contatore variabile a 3 elementi i cui
insiemi quozienti siano quelli modulo 5, 7 e 3 una possibile
configurazione degli elementi sarebbe (4,4,2). Un incremento
determinerebbe nel contatore il cambiamento della configurazione a
(4,5,0) in quanto l'aggiunta di 1 all'elemento meno significativo
causa un riporto che deve andare a trasferirsi all'elemento
successivo.

L'utilizzo del contatore variabile all'interno del calcolo dell'ordine
degli assegnamenti consiste nell'impiego di un contatore per riuscire
ad individuare senza ripetizioni tutti i flip peggiorativi di una
configurazione. Per fare ciò si stabilisce un ordine arbitrario a
tutti i valori dei domini di ciascuna variabile e ad ogni elemento del
contatore si associa una variabile ovvero un'insieme quoziente modulo
la cardinalità del dominio di tale variabile. Per evitare le
ripetizioni è sufficiente considerare una configurazione $i$ di un
contatore e calcolare tutte e sole le configurazioni che si ottengono
da $i$ modificando un solo elemento del contatore e che siano
\textit{strettamente maggiori di $i$}.

Ad esempio, considerando il contatore introdotto in precedenza, data
la configurazione $i$=(3,1,0), allora la configurazioni strettamente
maggiori diverse per al più un elemento sono: (3,1,1), (3,1,2),
(3,2,0), (3,3,0), (3,4,0), (3,5,0), (3,6,0), (4,1,0).

Stabilito l'ordine degli elementi per i domini di ogni variabile, dato
un contatore è immediato risalire a quale sia l'assegnamento delle
variabili che è rappresentata dal contatore. Pertanto data una
configurazione di contatore $i$ e le configurazioni strettamente
maggiori di $i$ ottenute con il procedimento precedentemente
descritto, si ottengono un assegnamento di variabili e tutti gli
assegnamenti a distanza 1. Usando i compartori descritti alla sezione
\ref{sect:comparatori} è immediato calcolare quale sia la relazione
esistente di preferenza tra gli assegnamenti.

L'algoritmo per il calcolo degli ordini degli assegnamenti è quindi il
seguente:
\begin{enumerate}
\item \textbf{Inizializzazione:} Si stabilisca un ordine per i valori
  di ciascuna variabile, si estraggano i comparatori dagli ordini di
  ogni variabile, si inizializzi un contatore variabile $c$ ad
  elementi pari alla cardinalità delle variabili, si inizializzi un
  grafo orientato $G$.
\item \textbf{Ciclo di calcolo:} Fino a che $c$ non è arrivato alla
  configurazione massima:
  \begin{enumerate}
  \item Si calcoli l'insieme delle configurazioni strettamente
    maggiori di $c$ distanti 1.
  \item Per ogni configurazione ottenuta al punto (a) si usi il
    comparatore associato all'unica variabile modificata per
    determinare quale delle configurazioni sia preferita e si segnali
    in $G$ la preferenza.
  \item Si incrementi $c$.
  \end{enumerate}
\end{enumerate}

\subsubsection{Complessità}
La complessità dell'algoritmo precedentemente descritto, ipotizzando
che si utilizzi la rappresentazione dei nodi interni degli ordini a
tabella hash, è esattamente pari a $O(e + v)$ in quanto:
\begin{itemize}
\item Tutti i passi di inizializzazione sono costanti e svolti in
  tempo $O(1)$.
\item Il numero di iterazioni del ciclo di calcolo è esattamente
  $O(v)$.
\item Il numero totale di configurazioni strettamente maggiori di $c$
  e distanti 1 nella totalità dell'esecuzione è $O(e)$ e il loro
  calcolo è a tempo costante, quindi il costo totale del passo (a) in
  tutta l'esecuzione è $O(e)$.
\item Con la tabella hash il costo per trovare l'ordine interessato è
  $O(1)$ quindi il costo complessivo del passo (b) in tutta
  l'esecuzione è $O(e)$ in quanto il confronto tra due valori trovato
  l'ordine è svolto in tempo costante.
\item L'incremento del contatore $c$ è realizzabile in tempo
  ammortizzato costante quindi il costo complessivo del passo (c) per l'intera
  esecuzione è $O(v)$.
\end{itemize}

Complessivamente quindi si ha $O(1) + O(e) + O(v)= O(e+v)$.
