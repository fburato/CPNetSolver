\subsection{Ricerca delle soluzioni}
La ricerca delle soluzioni della CP-Net è fatta attraverso un
algoritmo di tipo Backtracking+Forward Checking modificato per trovare
tutte le soluzioni ottime e non solo una soluzione.

L'algoritmo seguito per risolvere un problema a $m$ variabili è quindi
il seguente:
\begin{enumerate}
\item \textbf{Inizializzazione:} Si raccolgano tutti i domini di tutte
  le variabili del problema e tutti i vincoli ottenuti dalla
  conversione della CP-Net in CSP. Ciò consente di ottenere tutte le
  tuple ammesse dagli assegnamenti delle variabili in un ordine.
\item \textbf{Partenza:} Si esegua la ricerca in profondità partendo
  dal livello 0 avendo come vettore di supporto un vettore vuoto.
\item \textbf{Ricerca in profondità di livello i-esimo e vettore di
    supporto $w$:} Se $i=m$ allora $w$ contiene una soluzione alla
  CP-Net che viene segnalata, altrimenti se il dominio i-esimo è vuoto
  allora vuol dire che l'assegnamento precedente ha causato un
  inconsistenza e la ricerca a questo punto termina, altrimenti per
  ogni valore $v$ del dominio i-esimo:
  \begin{enumerate}
  \item Si riducano i vincoli attuali rispetto a $v$ ovvero si
    ottengono tutte le tuple che ammettono la variabile i-esima
    associata al valore $v$.
  \item Si intersechino i domini correnti con le proiezioni dei vincoli
    ovvero si ottengano tutti e soli i valori ammessi dai vincoli per
    ciascuna variabile avendo ridotto i domini rispetto a $v$.
  \item Si assegni $w[i]=v$.
  \item Si esegua la ricerca in profondità di livello i+1 con il nuovo
    w come vettore di supporto.
  \end{enumerate}
\end{enumerate}