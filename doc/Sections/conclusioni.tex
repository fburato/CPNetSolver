\section{Conclusioni}
CPNetSolver rappresenta un utile strumento per riuscire a risolvere e
visualizzare concretamente una CP-Net, sia essa ciclica o aciclica. Un
ovvio campo di applicazione per tale programma è evidentemente quello
didattico, in quanto la possibilità di modificare con una semplice
grammatica la definizione di una CP-Net e vedere immediatamente quali
sono le conseguenze sul grafo degli ordini degli assegnamenti e sulle
soluzioni, è uno strumento semplice ma estremamente utile per capire
concetti come la semantica dei flip peggiorativi e la necessità degli
ordini completi.

Nonostante si tratti di un'applicazione didattica è stata prestata
molta attenzione ai problemi computazionali e di rappresentazione
degli oggetti del dominio. Inoltre l'impiego di Scala e di uno stile
di programmazione funzionale potrebbe consentire, come ottimizzazione
futura, la possibile parallelizzazione degli algoritmi di calcolo
delle soluzioni ottime e di rappresentazione del grafo degli ordini
degli assegnamenti.
